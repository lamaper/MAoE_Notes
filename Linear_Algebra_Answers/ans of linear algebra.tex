\documentclass{ctexart}
\usepackage{amsmath, amsthm, amssymb, graphicx, geometry, arydshln, mdframed}
\geometry{left=2.54cm , right=2.54cm , top=3.17cm , bottom=3.17cm}

<<<<<<< HEAD
\title{Linear Algebra Answers (Chapter3)}
=======
\title{Linear Algebra Answers}
>>>>>>> ff75dda422ddcf940f668f37ca6e0754a76074f4
\author{Omniporent-ys}
\date{\today}

\begin{document}
\maketitle
    \paragraph{1}
        \begin{mdframed}
        3$\vec{\alpha}$ + 4$\vec{\beta}$
        = 3$\begin{bmatrix}
            1 \\
            2 \\
            3 \\
            4\\           
            \end{bmatrix}$
            + 4$\begin{bmatrix}
            0 \\
            2 \\
            4 \\
            6 \\           
            \end{bmatrix}$
            = $\begin{bmatrix}
            3 \\
            6 \\
            9 \\
            12 \\           
            \end{bmatrix}$
            +
            $\begin{bmatrix}
            0 \\
            8 \\
            16 \\
            24 \\           
            \end{bmatrix}$
            = 
            $\begin{bmatrix}
            3 \\
            14 \\
            25 \\
            36 \\           
            \end{bmatrix}$
        \end{mdframed}


    \paragraph{2}
        \begin{mdframed}
            2$\vec{\alpha}_1$ + 2$\vec{\beta}$ + 3$\vec{\alpha}_2$ - 3$\vec{\beta}$
                = 2$\vec{\alpha}_3$ + 2$\vec{\beta}$
                \\$\vec{\beta}$ = $\dfrac{2}{3} \vec{\alpha}_1$ + $\vec{\alpha}_2$ - $\dfrac{2}{3} \vec{\alpha}_3$
                = 
                $\renewcommand{\arraystretch}{1.7}
                \begin{bmatrix}
                ~\dfrac{5}{3}\\
                ~\dfrac{7}{3}\\
                ~\dfrac{5}{3}\\
                -\dfrac{4}{3}\\
                \end{bmatrix}$            
        \end{mdframed}


    \paragraph{3}
        \begin{mdframed}
            取\\
            $\vec{\alpha}$ = 
                $\begin{bmatrix}
                    x_1,x_2,\cdots,x_n\\
                \end{bmatrix}^{T}$ 
            $\in$ $\mathbf{V}$
            \\ $\vec{\beta}$ = 
            $\begin{bmatrix}
                y_1,y_2,\cdots,y_n\\
            \end{bmatrix}^T$ $\in$ $\mathbf{V}$\\
            则有\\
            $\vec{\alpha}+\vec{\beta}$ = 
                $\begin{bmatrix}
                    x_1+y_1 , x_2+y_2 , \cdots , x_n+y_n\\
                \end{bmatrix}^{T}$ 
            $\in$ $\mathbf{V}$\\
            设\\
            $x_i - a_i = c_x , y_i - a_i = c_y$,
                $x_i + y_i - a = c_{xy}$\\
            则有\\
            $a_i = c_{xy}-c_x-c_y$,\\
            即$a_i$的值与i无关\\
            这说明\\
            $a_1=a_2=\cdots=a_n$\\ \\
            $k\vec{\alpha}\in\mathbf{V}$请自行验证
        \end{mdframed}


    \paragraph{4}
        \begin{mdframed}
            (1)\\
            取\\
            $\vec{\alpha}$=
            $\begin{bmatrix}
                1,1
            \end{bmatrix}^{T}$
            $\in \mathbf{V_1}$\\
            则\\
            $(-1)\vec{\alpha}$=
            $\begin{bmatrix}
                -1,-1
            \end{bmatrix}^{T}$
            $\notin \mathbf{V_1}$\\
            \\
            (2)\\
            取\\
            $\vec{\alpha_1}$=
            $\begin{bmatrix}
                1,0
            \end{bmatrix}^{T}$
            $\in \mathbf{V_2}$\\
            $\vec{\alpha_2}$=
            $\begin{bmatrix}
                0,-1
            \end{bmatrix}^{T}$
            $\in \mathbf{V_2}$\\
            则\\
            $\vec{\alpha_1}+\vec{\alpha_2}$=
            $\begin{bmatrix}
                1,-1
            \end{bmatrix}^{T}$
            $\notin\mathbf{V_2}$\\
            \\
            (3)\\
            取\\
            $\vec{\alpha_1}$=
            $\begin{bmatrix}
                1,0
            \end{bmatrix}^{T}$
            $\in \mathbf{V_3}$\\
            $\vec{\alpha_2}$=
            $\begin{bmatrix}
                0,1
            \end{bmatrix}^{T}$
            $\in \mathbf{V_3}$\\
            则\\
            $\vec{\alpha_1}+\vec{\alpha_2}$=
            $\begin{bmatrix}
                1,1
            \end{bmatrix}^{T}$
            $\notin\mathbf{V_3}$\\

        \end{mdframed}


    \paragraph{5}
        \begin{mdframed}
            取\\
            $\vec{\alpha}$=
            $\begin{bmatrix}
                0,1,\dots,1,1
            \end{bmatrix}^{T}$
            $\in\mathbf{W}$\\
            $\vec{\beta}$=
            $\begin{bmatrix}
                1,1,\dots,1,0
            \end{bmatrix}^{T}$
            $\in\mathbf{W}$\\
            则\\
            $\vec{\alpha}+\vec{\beta}=$
            $\begin{bmatrix}
                1,2,\dots,2,1
            \end{bmatrix}^{T}$
            $\notin\mathbf{W}$\\
            这说明$\mathbf{W}$不构成$\mathbf{R}$上的向量空间
        \end{mdframed}
            

    \paragraph{6}
    \begin{mdframed}
        (1)\\
        取\\
        $\vec{\alpha_1}=$
        $\begin{bmatrix}
            3x_1+2x_2,x_1,x_2
        \end{bmatrix}^{T}$
        $\in \mathbf{W}$\\
        $\vec{\alpha_2}=$
        $\begin{bmatrix}
            3y_1+2y_2,y_1,y_2
        \end{bmatrix}^{T}$
        $\in \mathbf{W}$\\
        则\\
        $\vec{\alpha_1}+\vec{\alpha_2}=
        \begin{bmatrix}
            3(x_1+y_1)+2(x_2+y_2),x_1+y_1,x_2+y_2
        \end{bmatrix}^{T}
        \in \mathbf{W}$\\
        $k\vec{\alpha_1}=
        \begin{bmatrix}
            3(k(x_1))+2(k(x_2)),k(x_1),k(x_2)
        \end{bmatrix}^{T}
        \in \mathbf{W}$\\
        说明$\mathbf{W}$为$\mathbf{R}_3$的一个子空间\\
        (2)\\
        观察知,取\\
        $\vec{\beta}=
        \begin{bmatrix}
            3,1,0
        \end{bmatrix}^{T}$\\
        $\vec{\gamma}=
        \begin{bmatrix}
            2,0,1
        \end{bmatrix}^{T}$\\
        则\\
        %$\mathbf{W}=Span{\vec{\beta},\vec{\gamma}}$
        $\mathbf{W} = \text{Span}\{\vec{\beta}, \vec{\gamma}\}$
    \end{mdframed}
\end{document}\documentclass{ctexart}
            
\end{document}